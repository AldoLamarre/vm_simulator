\documentclass{article}

\usepackage[utf8]{inputenc}
\usepackage[letterpaper, margin=1.2in]{geometry}
\usepackage{listings}

\title{Travail pratique \#3 - IFT-2245}
\author{Aldo Lamarre et Frédéric Hamel }

\begin{document}

\maketitle

%%allo comment ça va? Plutôt bien le tp n'est pas énervant cette fois
%% ¡¡ REMPLIR ICI !!

\section*{TLB}
Nous implémentons les algorithmes FIFO et seconde chance, car ils étaient plus 
simple à implémenter. Seconde chance étant une amélioration de LRU qui elle 
même est une amélioration de FIFO, c'était logique une fois FIFO implémenter 
de faire sont amélioration en le transformant en seconde chance pour le 
deuxième algorithme.

\subsection{FIFO}
\paragraph{Avantages}
Accès rapide à la prochaine victime.
Organisation linéaire des éléments.
Faciliter d'implémentation. 

\paragraph{Inconvénients}
Anomalie de Belady, dans certains cas il est possible qu'augmenter le nombre de frames 
augmente le nombre de page faults. Dans le pire cas, il y a juste des tlb miss.
Étant donné que la prochaine page qui pourrait être lu est toujours remplacer 
par l'algorithme FIFO. 

\subsection{CLOCK ou seconde chance}
\paragraph{Avantages}
Implémentation plus simple et complexité en temps plus petite que l'algorithme de LRU.
CLOCK est bonne approximation de LRU étant donnée qu'il utilise moins de méta-données que LRU.
Donne une chance au page référencée contrairement à FIFO.

\paragraph{Inconvénients}
Le pire cas de FIFO existe encore. Utilisation de méta-données explicites contrairement
à FIFO ou les méta-données sont implicites. 

\section*{Test défaut}
\subsection*{Random io}
\paragraph{FIFO}
\begin{lstlisting}
tlb hits:   16328
tlb miss:   246158
tlb hit ratio:   0.033166
page found: 256534
page fault: 5952
page fault ratio:   0.022675
\end{lstlisting}

\paragraph{Seconde Chance}
\begin{lstlisting}
tlb hits:   16305
tlb miss:   246181
tlb hit ratio:   0.033116
page found: 256776
page fault: 5710
page fault ratio:   0.021754
\end{lstlisting}
\paragraph{Comparaison}
Il y a plus de tlb ir dans le premier, mais plus de page found dans le second.
Cela s'explique par le fait que l'algorithme de remplacement de page utilise dans
les deux cas utilise les données contenues dans le TLB. Même s'il y a légèrement moins
de tlb hits dans le cas de second chance (CLOCK), le nombre de page faults a diminué de façon
significative étant donnée que les pages visitées ne sont pas remis immédiatement évincée.

\subsection*{Écriture en série (Mars attack)}
\paragraph{FIFO}
\begin{lstlisting}
tlb hits:   65280
tlb miss:   256
tlb hit ratio:   127.500000
page found: 65280
page fault: 256
page fault ratio:   0.003906
\end{lstlisting}

\paragraph{Seconde Chance}
\begin{lstlisting}
tlb hits:   65280
tlb miss:   256
tlb hit ratio:   127.500000
page found: 65280
page fault: 256
page fault ratio:   0.003906
\end{lstlisting}
\paragraph{Comparaison}
Dans ce cas précis, l'algorithme CLOCK se comporte exactement comme FIFO.
C'est le pire cas de CLOCK.

\section*{Test Random io avec NUM\_FRAMES = 128}
\subsection*{TLB\_NUM\_ENTRIES = 16 }
\paragraph{FIFO}
\begin{lstlisting}
tlb hits:   16328
tlb miss:   246158
tlb hit ratio:   0.033166
page found: 256534
page fault: 5952
page fault ratio:   0.022675
\end{lstlisting}
\paragraph{Seconde Chance}
\begin{lstlisting}
tlb hits:   16305
tlb miss:   246181
tlb hit ratio:   0.033116
page found: 256776
page fault: 5710
page fault ratio:   0.021754
\end{lstlisting}

\subsection*{TLB\_NUM\_ENTRIES = 32}
\paragraph{FIFO}
\begin{lstlisting}
tlb hits:   32783
tlb miss:   229703
tlb hit ratio:   0.071360
page found: 260086
page fault: 2400
page fault ratio:   0.009143
\end{lstlisting}
\paragraph{Seconde Chance}
\begin{lstlisting}
tlb hits:   32849
tlb miss:   229637
tlb hit ratio:   0.071524
page found: 260046
page fault: 2440
page fault ratio:   0.009296


\end{lstlisting}
\subsection*{TLB\_NUM\_ENTRIES = 64}
\paragraph{FIFO}
\begin{lstlisting}
tlb hits:   65647
tlb miss:   196839
tlb hit ratio:   0.166753
page found: 261402
page fault: 1084
page fault ratio:   0.004130


\end{lstlisting}
\paragraph{Seconde Chance}
\begin{lstlisting}
tlb hits:   65553
tlb miss:   196933
tlb hit ratio:   0.166435
page found: 261509
page fault: 977
page fault ratio:   0.003722
\end{lstlisting}

\paragraph{Comparaison}
Nous remarquons qu'augmenter le nombre d'entrées du TLB augmente le nombre de hits.
Ce teste ne montre pas les symptômes de l'anomalie de Belady. Le nombre de page fault
diminue inversement proportionnellement au nombre d'entrées dans le TLB. Bref, plus le
TLB a d'entrée moins il y a de page fault.

\subsection*{tbl = 256 et frame = 256}
\paragraph{FIFO}
\begin{lstlisting}
tlb hits:   262230
tlb miss:   256
tlb hit ratio:   512.167969
page found: 262230
page fault: 256
page fault ratio:   0.000975
\end{lstlisting}

\paragraph{Seconde Chance}
\begin{lstlisting}
tlb hits:   262230
tlb miss:   256
tlb hit ratio:   512.167969
page found: 262230
page fault: 256
page fault ratio:   0.000975
\end{lstlisting}
\paragraph{Comparaison}
Dans le cas ou la taille du TLB est équivalente au nombre de frames,
on remarque un équivalence direct entre les deux algorithme.

\section*{Test Random io avec NUM\_FRAMES = 64}
\subsection*{TLB\_NUM\_ENTRIES = 16 }

\paragraph{FIFO}
\begin{lstlisting}
tlb hits:   16328
tlb miss:   246158
tlb hit ratio:   0.033166
page found: 210480
page fault: 52006
page fault ratio:   0.198129
\end{lstlisting}

\paragraph{Seconde Chance}
\begin{lstlisting}
tlb hits:   16305
tlb miss:   246181
tlb hit ratio:   0.033116
page found: 215993
page fault: 46493
page fault ratio:   0.177126
\end{lstlisting}

\subsection*{TLB\_NUM\_ENTRIES = 32}
\paragraph{FIFO}
\begin{lstlisting}
tlb hits:   32849
tlb miss:   229637
tlb hit ratio:   0.071524
page found: 248648
page fault: 13838
page fault ratio:   0.052719
\end{lstlisting}

\paragraph{Seconde Chance}
\begin{lstlisting}
tlb hits:   32783
tlb miss:   229703
tlb hit ratio:   0.071360
page found: 244586
page fault: 17900
page fault ratio:   0.068194
\end{lstlisting}

\subsection*{TLB\_NUM\_ENTRIES = 64}
\paragraph{FIFO}
\begin{lstlisting}
tlb hits:   65647
tlb miss:   196839
tlb hit ratio:   0.166753
page found: 259039
page fault: 3447
page fault ratio:   0.013132
\end{lstlisting}

\paragraph{Seconde Chance}
\begin{lstlisting}
tlb hits:   65553
tlb miss:   196933
tlb hit ratio:   0.166435
page found: 259194
page fault: 3292
page fault ratio:   0.012542
\end{lstlisting}
\paragraph{Comparaison}
Il y a une augmentation du nombre de page faults plus on diminue le nombre de frame.
Cela montre la différence du nombre de page faults des 2 algorithme FIFO et CLOCK.
Le ratio de tlb hit a aussi augmenter avec la taille du TLB. Le nombre de page fult diminue
avec la taille du TLB. 

\section*{Comparaison des algorithme avec NUM\_FRAMES = NUM\_PAGES = 256}
\subsection*{FIFO}
\begin{lstlisting}
tlb size:   16
tlb hits:   16328
tlb miss:   246158
tlb hit ratio:   0.033166
page found: 262230
page fault: 256
page fault ratio:   0.000975
\end{lstlisting}
\begin{lstlisting}
tlb size:   32
tlb hits:   32783
tlb miss:   229703
tlb hit ratio:   0.071360
page found: 262230
page fault: 256
page fault ratio:   0.000975
\end{lstlisting}

\subsection*{Second chance CLOCK}
\begin{lstlisting}
tlb size:   16
tlb hits:   16305
tlb miss:   246181
tlb hit ratio:   0.033116
page found: 262230
page fault: 256
page fault ratio:   0.000975
\end{lstlisting}
\begin{lstlisting}
tlb size:   32
tlb hits:   32849
tlb miss:   229637
tlb hit ratio:   0.071524
page found: 262230
page fault: 256
page fault ratio:   0.000975
\end{lstlisting}
\paragraph{Comparaison}
Seconde chance se retrouve en dessous de FIFO dans le cas ou le TLB est de 16 entrées.
Par contre, seconde chance(CLOCK) est au dessus de FIFO avec une taille TLB de 32.
Bref, c'est deux algorithme on beaucoup des similarités.
\end{document}
