\documentclass{article}

\usepackage[utf8]{inputenc}

\title{Travail pratique \#3 - IFT-2245}
\author{Aldo Lamarre et Frédéric Hamel }

\begin{document}

\maketitle

%%allo comment ça va? plutôt bien le tp n'est pas énervant cette fois
%% ¡¡ REMPLIR ICI !!

\section*{TLB}
Nous implémentons les algorithmes FIFO et seconde chance, car ils étaient plus 
simple à implémenter. Seconde chance étant une amélioration de LRU qui elle 
même est une amélioration de FIFO, c'était logique une fois FIFO implémenter 
de faire sont amélioration en le transformant en seconde chance pour le 
deuxième algorithme.
\subsection{FIFO}
\paragraph{Avantages}
Accès rapide à la prochaine victime.
Organisation linéaire des éléments.
Faciliter d'implémentation. 

\paragraph{Inconvénients}
Anomalie de Belady, dans certains cas il est possible qu'augmenter les frames 
augmentent le nombre de page faults.Dans le pire cas, il y a juste des tlb miss.
Étant donné que la prochaine page qui pourrait être lu est toujours remplacer 
par l'algorithme FIFO. 

\subsection{CLOCK ou seconde chance}
\paragraph{Avantages}
Implémentation plus simple et complexifier en temps plus petite que l'algorithme de LRU.
CLOCK est bonne approximation de LRU en utilisant moins de méta-données que LRU.
Donne une chance au page référencée contrairement à FIFO.

\paragraph{Inconvénients}
Le pire cas de FIFO existe encore.Utilisation de méta-données explicites contrairement
 à FIFO ou les méta-données sont implicites. 


\section*{}

\section*{}

\end{document}
